% Generated by Sphinx.
\def\sphinxdocclass{report}
\documentclass[letterpaper,10pt,english]{sphinxmanual}
\usepackage[utf8]{inputenc}
\DeclareUnicodeCharacter{00A0}{\nobreakspace}
\usepackage{cmap}
\usepackage[T1]{fontenc}
\usepackage{babel}
\usepackage{times}
\usepackage[Bjarne]{fncychap}
\usepackage{longtable}
\usepackage{sphinx}
\usepackage{multirow}


\title{PyOEIS Documentation}
\date{January 17, 2015}
\release{1.0}
\author{Dylan Evans}
\newcommand{\sphinxlogo}{}
\renewcommand{\releasename}{Release}
\makeindex

\makeatletter
\def\PYG@reset{\let\PYG@it=\relax \let\PYG@bf=\relax%
    \let\PYG@ul=\relax \let\PYG@tc=\relax%
    \let\PYG@bc=\relax \let\PYG@ff=\relax}
\def\PYG@tok#1{\csname PYG@tok@#1\endcsname}
\def\PYG@toks#1+{\ifx\relax#1\empty\else%
    \PYG@tok{#1}\expandafter\PYG@toks\fi}
\def\PYG@do#1{\PYG@bc{\PYG@tc{\PYG@ul{%
    \PYG@it{\PYG@bf{\PYG@ff{#1}}}}}}}
\def\PYG#1#2{\PYG@reset\PYG@toks#1+\relax+\PYG@do{#2}}

\expandafter\def\csname PYG@tok@gd\endcsname{\def\PYG@tc##1{\textcolor[rgb]{0.63,0.00,0.00}{##1}}}
\expandafter\def\csname PYG@tok@gu\endcsname{\let\PYG@bf=\textbf\def\PYG@tc##1{\textcolor[rgb]{0.50,0.00,0.50}{##1}}}
\expandafter\def\csname PYG@tok@gt\endcsname{\def\PYG@tc##1{\textcolor[rgb]{0.00,0.27,0.87}{##1}}}
\expandafter\def\csname PYG@tok@gs\endcsname{\let\PYG@bf=\textbf}
\expandafter\def\csname PYG@tok@gr\endcsname{\def\PYG@tc##1{\textcolor[rgb]{1.00,0.00,0.00}{##1}}}
\expandafter\def\csname PYG@tok@cm\endcsname{\let\PYG@it=\textit\def\PYG@tc##1{\textcolor[rgb]{0.25,0.50,0.56}{##1}}}
\expandafter\def\csname PYG@tok@vg\endcsname{\def\PYG@tc##1{\textcolor[rgb]{0.73,0.38,0.84}{##1}}}
\expandafter\def\csname PYG@tok@m\endcsname{\def\PYG@tc##1{\textcolor[rgb]{0.13,0.50,0.31}{##1}}}
\expandafter\def\csname PYG@tok@mh\endcsname{\def\PYG@tc##1{\textcolor[rgb]{0.13,0.50,0.31}{##1}}}
\expandafter\def\csname PYG@tok@cs\endcsname{\def\PYG@tc##1{\textcolor[rgb]{0.25,0.50,0.56}{##1}}\def\PYG@bc##1{\setlength{\fboxsep}{0pt}\colorbox[rgb]{1.00,0.94,0.94}{\strut ##1}}}
\expandafter\def\csname PYG@tok@ge\endcsname{\let\PYG@it=\textit}
\expandafter\def\csname PYG@tok@vc\endcsname{\def\PYG@tc##1{\textcolor[rgb]{0.73,0.38,0.84}{##1}}}
\expandafter\def\csname PYG@tok@il\endcsname{\def\PYG@tc##1{\textcolor[rgb]{0.13,0.50,0.31}{##1}}}
\expandafter\def\csname PYG@tok@go\endcsname{\def\PYG@tc##1{\textcolor[rgb]{0.20,0.20,0.20}{##1}}}
\expandafter\def\csname PYG@tok@cp\endcsname{\def\PYG@tc##1{\textcolor[rgb]{0.00,0.44,0.13}{##1}}}
\expandafter\def\csname PYG@tok@gi\endcsname{\def\PYG@tc##1{\textcolor[rgb]{0.00,0.63,0.00}{##1}}}
\expandafter\def\csname PYG@tok@gh\endcsname{\let\PYG@bf=\textbf\def\PYG@tc##1{\textcolor[rgb]{0.00,0.00,0.50}{##1}}}
\expandafter\def\csname PYG@tok@ni\endcsname{\let\PYG@bf=\textbf\def\PYG@tc##1{\textcolor[rgb]{0.84,0.33,0.22}{##1}}}
\expandafter\def\csname PYG@tok@nl\endcsname{\let\PYG@bf=\textbf\def\PYG@tc##1{\textcolor[rgb]{0.00,0.13,0.44}{##1}}}
\expandafter\def\csname PYG@tok@nn\endcsname{\let\PYG@bf=\textbf\def\PYG@tc##1{\textcolor[rgb]{0.05,0.52,0.71}{##1}}}
\expandafter\def\csname PYG@tok@no\endcsname{\def\PYG@tc##1{\textcolor[rgb]{0.38,0.68,0.84}{##1}}}
\expandafter\def\csname PYG@tok@na\endcsname{\def\PYG@tc##1{\textcolor[rgb]{0.25,0.44,0.63}{##1}}}
\expandafter\def\csname PYG@tok@nb\endcsname{\def\PYG@tc##1{\textcolor[rgb]{0.00,0.44,0.13}{##1}}}
\expandafter\def\csname PYG@tok@nc\endcsname{\let\PYG@bf=\textbf\def\PYG@tc##1{\textcolor[rgb]{0.05,0.52,0.71}{##1}}}
\expandafter\def\csname PYG@tok@nd\endcsname{\let\PYG@bf=\textbf\def\PYG@tc##1{\textcolor[rgb]{0.33,0.33,0.33}{##1}}}
\expandafter\def\csname PYG@tok@ne\endcsname{\def\PYG@tc##1{\textcolor[rgb]{0.00,0.44,0.13}{##1}}}
\expandafter\def\csname PYG@tok@nf\endcsname{\def\PYG@tc##1{\textcolor[rgb]{0.02,0.16,0.49}{##1}}}
\expandafter\def\csname PYG@tok@si\endcsname{\let\PYG@it=\textit\def\PYG@tc##1{\textcolor[rgb]{0.44,0.63,0.82}{##1}}}
\expandafter\def\csname PYG@tok@s2\endcsname{\def\PYG@tc##1{\textcolor[rgb]{0.25,0.44,0.63}{##1}}}
\expandafter\def\csname PYG@tok@vi\endcsname{\def\PYG@tc##1{\textcolor[rgb]{0.73,0.38,0.84}{##1}}}
\expandafter\def\csname PYG@tok@nt\endcsname{\let\PYG@bf=\textbf\def\PYG@tc##1{\textcolor[rgb]{0.02,0.16,0.45}{##1}}}
\expandafter\def\csname PYG@tok@nv\endcsname{\def\PYG@tc##1{\textcolor[rgb]{0.73,0.38,0.84}{##1}}}
\expandafter\def\csname PYG@tok@s1\endcsname{\def\PYG@tc##1{\textcolor[rgb]{0.25,0.44,0.63}{##1}}}
\expandafter\def\csname PYG@tok@gp\endcsname{\let\PYG@bf=\textbf\def\PYG@tc##1{\textcolor[rgb]{0.78,0.36,0.04}{##1}}}
\expandafter\def\csname PYG@tok@sh\endcsname{\def\PYG@tc##1{\textcolor[rgb]{0.25,0.44,0.63}{##1}}}
\expandafter\def\csname PYG@tok@ow\endcsname{\let\PYG@bf=\textbf\def\PYG@tc##1{\textcolor[rgb]{0.00,0.44,0.13}{##1}}}
\expandafter\def\csname PYG@tok@sx\endcsname{\def\PYG@tc##1{\textcolor[rgb]{0.78,0.36,0.04}{##1}}}
\expandafter\def\csname PYG@tok@bp\endcsname{\def\PYG@tc##1{\textcolor[rgb]{0.00,0.44,0.13}{##1}}}
\expandafter\def\csname PYG@tok@c1\endcsname{\let\PYG@it=\textit\def\PYG@tc##1{\textcolor[rgb]{0.25,0.50,0.56}{##1}}}
\expandafter\def\csname PYG@tok@kc\endcsname{\let\PYG@bf=\textbf\def\PYG@tc##1{\textcolor[rgb]{0.00,0.44,0.13}{##1}}}
\expandafter\def\csname PYG@tok@c\endcsname{\let\PYG@it=\textit\def\PYG@tc##1{\textcolor[rgb]{0.25,0.50,0.56}{##1}}}
\expandafter\def\csname PYG@tok@mf\endcsname{\def\PYG@tc##1{\textcolor[rgb]{0.13,0.50,0.31}{##1}}}
\expandafter\def\csname PYG@tok@err\endcsname{\def\PYG@bc##1{\setlength{\fboxsep}{0pt}\fcolorbox[rgb]{1.00,0.00,0.00}{1,1,1}{\strut ##1}}}
\expandafter\def\csname PYG@tok@mb\endcsname{\def\PYG@tc##1{\textcolor[rgb]{0.13,0.50,0.31}{##1}}}
\expandafter\def\csname PYG@tok@ss\endcsname{\def\PYG@tc##1{\textcolor[rgb]{0.32,0.47,0.09}{##1}}}
\expandafter\def\csname PYG@tok@sr\endcsname{\def\PYG@tc##1{\textcolor[rgb]{0.14,0.33,0.53}{##1}}}
\expandafter\def\csname PYG@tok@mo\endcsname{\def\PYG@tc##1{\textcolor[rgb]{0.13,0.50,0.31}{##1}}}
\expandafter\def\csname PYG@tok@kd\endcsname{\let\PYG@bf=\textbf\def\PYG@tc##1{\textcolor[rgb]{0.00,0.44,0.13}{##1}}}
\expandafter\def\csname PYG@tok@mi\endcsname{\def\PYG@tc##1{\textcolor[rgb]{0.13,0.50,0.31}{##1}}}
\expandafter\def\csname PYG@tok@kn\endcsname{\let\PYG@bf=\textbf\def\PYG@tc##1{\textcolor[rgb]{0.00,0.44,0.13}{##1}}}
\expandafter\def\csname PYG@tok@o\endcsname{\def\PYG@tc##1{\textcolor[rgb]{0.40,0.40,0.40}{##1}}}
\expandafter\def\csname PYG@tok@kr\endcsname{\let\PYG@bf=\textbf\def\PYG@tc##1{\textcolor[rgb]{0.00,0.44,0.13}{##1}}}
\expandafter\def\csname PYG@tok@s\endcsname{\def\PYG@tc##1{\textcolor[rgb]{0.25,0.44,0.63}{##1}}}
\expandafter\def\csname PYG@tok@kp\endcsname{\def\PYG@tc##1{\textcolor[rgb]{0.00,0.44,0.13}{##1}}}
\expandafter\def\csname PYG@tok@w\endcsname{\def\PYG@tc##1{\textcolor[rgb]{0.73,0.73,0.73}{##1}}}
\expandafter\def\csname PYG@tok@kt\endcsname{\def\PYG@tc##1{\textcolor[rgb]{0.56,0.13,0.00}{##1}}}
\expandafter\def\csname PYG@tok@sc\endcsname{\def\PYG@tc##1{\textcolor[rgb]{0.25,0.44,0.63}{##1}}}
\expandafter\def\csname PYG@tok@sb\endcsname{\def\PYG@tc##1{\textcolor[rgb]{0.25,0.44,0.63}{##1}}}
\expandafter\def\csname PYG@tok@k\endcsname{\let\PYG@bf=\textbf\def\PYG@tc##1{\textcolor[rgb]{0.00,0.44,0.13}{##1}}}
\expandafter\def\csname PYG@tok@se\endcsname{\let\PYG@bf=\textbf\def\PYG@tc##1{\textcolor[rgb]{0.25,0.44,0.63}{##1}}}
\expandafter\def\csname PYG@tok@sd\endcsname{\let\PYG@it=\textit\def\PYG@tc##1{\textcolor[rgb]{0.25,0.44,0.63}{##1}}}

\def\PYGZbs{\char`\\}
\def\PYGZus{\char`\_}
\def\PYGZob{\char`\{}
\def\PYGZcb{\char`\}}
\def\PYGZca{\char`\^}
\def\PYGZam{\char`\&}
\def\PYGZlt{\char`\<}
\def\PYGZgt{\char`\>}
\def\PYGZsh{\char`\#}
\def\PYGZpc{\char`\%}
\def\PYGZdl{\char`\$}
\def\PYGZhy{\char`\-}
\def\PYGZsq{\char`\'}
\def\PYGZdq{\char`\"}
\def\PYGZti{\char`\~}
% for compatibility with earlier versions
\def\PYGZat{@}
\def\PYGZlb{[}
\def\PYGZrb{]}
\makeatother

\renewcommand\PYGZsq{\textquotesingle}

\begin{document}

\maketitle
\tableofcontents
\phantomsection\label{index::doc}


Contents:


\chapter{Introduction}
\label{introduction:introduction}\label{introduction::doc}\label{introduction:welcome-to-pyoeis-s-documentation}
The \href{http://www.oeis.org}{Online Encyclopedia of Integer Sequences (OEIS)} is
a database of integer sequences. It contains many well-known sequences, such as
the primes and Fibonacci numbers as well as many more obscure sequences.

PyOEIS allows for the searching of the OEIS from within Python, as well as
allowing you to access sequence information through
{\hyperref[api:sequence.Sequence]{\code{Sequence}}} objects.


\section{OEIS Search Syntax and Other Useful Links}
\label{introduction:oeis-search-syntax-and-other-useful-links}
While PyOEIS provides many porcelain methods which allow you to search by
sequence ID, author etc., it is also possible to query the OEIS using its
usual search syntax (as you would on the website). It may also be useful to
understand how the OEIS operates for debugging purposes. The following links
may therefore be of use:
\begin{itemize}
\item {} 
\href{http://oeis.org/hints.html}{Hints and search syntax}

\item {} 
\href{http://oeis.org/eishelp1.html}{Explanation of the OEIS internal format (which PyOEIS is built upon)}

\item {} 
\href{http://oeis.org/eishelp2.html}{Explanation of the fields in each sequence entry}

\end{itemize}


\chapter{API}
\label{api:api}\label{api::doc}

\section{\texttt{OEISClient} objects}
\label{api:oeisclient-objects}\index{OEISClient (class in client)}

\begin{fulllineitems}
\phantomsection\label{api:client.OEISClient}\pysigline{\strong{class }\code{client.}\bfcode{OEISClient}}
Maintains a \href{http://docs.python-requests.org/en/latest/api/\#requests.Session}{\code{Session}} and contains
all methods for querying the OEIS.
\index{get\_by\_id() (OEISClient method)}

\begin{fulllineitems}
\phantomsection\label{api:OEISClient.get_by_id}\pysiglinewithargsret{\bfcode{get\_by\_id}}{\emph{id}}{}
Returns a {\hyperref[api:sequence.Sequence]{\code{Sequence}}} for the sequence with the
ID \emph{id}, or else raises {\hyperref[api:errors.NoResultsError]{\code{NoResultsError}}}.

\begin{notice}{note}{Note:}
On the OEIS website, IDs are displayed with an uppercase letter and 6
(for A IDs) or 4 (for M and N IDs) digits. However, this method does not
require an uppercase letter or leading zeros to be used.
\end{notice}

\end{fulllineitems}

\index{lookup\_by\_name() (OEISClient method)}

\begin{fulllineitems}
\phantomsection\label{api:OEISClient.lookup_by_name}\pysiglinewithargsret{\bfcode{lookup\_by\_name}}{\emph{name}, \emph{max\_seqs=10}}{}
Returns a list of at most \emph{max\_seqs} {\hyperref[api:sequence.Sequence]{\code{Sequence}}}
objects whose names contain \emph{query}.

\begin{notice}{note}{Note:}
Sequences are retrieved in sets of 10 and sequences are then removed if
necessary. So, there is no speed improvement between, for example, a
\emph{max\_seqs} of 10 and one of 15. This applies to all methods with a
\emph{max\_seqs} argument.
\end{notice}

\end{fulllineitems}

\index{lookup\_by\_author() (OEISClient method)}

\begin{fulllineitems}
\phantomsection\label{api:OEISClient.lookup_by_author}\pysiglinewithargsret{\bfcode{lookup\_by\_author}}{\emph{author}, \emph{max\_seqs=10}}{}
Returns a list of at most \emph{max\_seqs} {\hyperref[api:sequence.Sequence]{\code{Sequence}}}
objects whose authors contain \emph{query}.

\end{fulllineitems}

\index{lookup\_by() (client.OEISClient method)}

\begin{fulllineitems}
\phantomsection\label{api:client.OEISClient.lookup_by}\pysiglinewithargsret{\bfcode{lookup\_by}}{\emph{prefix}, \emph{query}, \emph{max\_seqs=10}, \emph{list\_func=False}}{}
If \emph{prefix} is \emph{``''}, search OEIS with string \emph{query},
otherwise use string `\emph{prefix}:\emph{query}`.

If \emph{list\_func} is true, return a list of at most
\emph{max\_seqs} {\hyperref[api:sequence.Sequence]{\code{Sequence}}} objects or else an
empty list if there are no results. If \emph{list\_func} is false, return
the first Sequence found, or else raise a
{\hyperref[api:errors.NoResultsError]{\code{NoResultsError}}}.

\end{fulllineitems}

\index{lookup\_by\_keywords() (client.OEISClient method)}

\begin{fulllineitems}
\phantomsection\label{api:client.OEISClient.lookup_by_keywords}\pysiglinewithargsret{\bfcode{lookup\_by\_keywords}}{\emph{keywords}}{}
Returns a list of at most \emph{max\_seqs}
{\hyperref[api:sequence.Sequence]{\code{Sequence}}} objects which are tagged with
\emph{keywords}.

\end{fulllineitems}

\index{lookup\_by\_terms() (client.OEISClient method)}

\begin{fulllineitems}
\phantomsection\label{api:client.OEISClient.lookup_by_terms}\pysiglinewithargsret{\bfcode{lookup\_by\_terms}}{\emph{terms}, \emph{**kwargs}}{}
Returns a list of at most \emph{max\_seqs}
{\hyperref[api:sequence.Sequence]{\code{Sequence}}} objects which contain \emph{terms}
anywhere within them. If none exist, returns an empty list. If
\emph{ordered} is false, terms may be in any order. If \emph{signed} is false,
terms may be positive or negative.

\end{fulllineitems}


\end{fulllineitems}



\section{\texttt{Sequence} objects}
\label{api:sequence-objects}

\subsection{Attributes}
\label{api:attributes}
\begin{tabulary}{\linewidth}{|L|L|}
\hline

id
 & 
The sequence's unique ID in the OEIS, as a string. Begins `A'.
\\

alt\_ids
 & 
Other IDs, as a list of strings beginning `M' and `N' which
the sequence carried in the ``The Encyclopedia of Integer
Sequences'', 1995 or the ``Handbook of Integer Sequences'', 1973,
respectively.
\\

unsigned
 & 
A list of terms in the sequence without any minus signs.
\\

signed
 & 
A list of terms in the sequence \emph{including} any minus signs.
\\

name
 & 
The name of the sequence, as a string.
\\

references
 & 
A list of references to the sequence.
\\

links
 & 
A list of links about the sequence.
\\

formulae
 & 
Formulae for generating the sequence, as a list of strings.
\\

cross\_references
 & 
Cross-references to the sequence from elsewhere in the OEIS,
as a list of strings.
\\

author
 & 
The author of the sequence entry, as a string.
\\

offset
 & 
The subscript of the first term and the position of the first
term whose modulus exceeds 1, as a tuple of two numbers.
\\

errors
 & 
Errors in the sequence entry, as a list of strings.
\\

examples
 & 
Examples to illustrate the sequence, as a list of strings.
\\

maple
 & 
Maple code to generate the sequence, as a string.
\\

mathematica
 & 
Mathematica code to generate the sequence, as a string.
\\

other\_programs
 & 
Code to generate the sequence in other programs/languages, as
a list of strings.
\\

keywords
 & 
The sequence's keywords, as a list of strings.
\\

comments
 & 
Comments on the sequence entry, as a list of strings.
\\
\hline\end{tabulary}


More information about the fields in a sequence entry can be found \href{http://oeis.org/eishelp2.html}{here}.


\subsection{Methods}
\label{api:methods}\index{Sequence (class in sequence)}

\begin{fulllineitems}
\phantomsection\label{api:sequence.Sequence}\pysigline{\strong{class }\code{sequence.}\bfcode{Sequence}}
Has attributes to contain information for each field of a sequence entry
in the OEIS and methods fot retrieving a certain number of the
sequence's signed or unsigned terms.
\index{signed() (sequence.Sequence method)}

\begin{fulllineitems}
\phantomsection\label{api:sequence.Sequence.signed}\pysiglinewithargsret{\bfcode{signed}}{\emph{n}}{}
Returns the first \emph{n} signed integers in the sequence.

\end{fulllineitems}

\index{unsigned() (sequence.Sequence method)}

\begin{fulllineitems}
\phantomsection\label{api:sequence.Sequence.unsigned}\pysiglinewithargsret{\bfcode{unsigned}}{\emph{n}}{}
Returns the first \emph{n} unsigned integers in the sequence.

\end{fulllineitems}


\end{fulllineitems}



\section{\texttt{Errors}}
\label{api:module-errors}\label{api:errors}\index{errors (module)}\index{InvalidQueryError}

\begin{fulllineitems}
\phantomsection\label{api:errors.InvalidQueryError}\pysiglinewithargsret{\strong{exception }\code{errors.}\bfcode{InvalidQueryError}}{\emph{response}}{}
Raised when a search is invalid according to the OEIS search syntax.

\end{fulllineitems}

\index{NoResultsError}

\begin{fulllineitems}
\phantomsection\label{api:errors.NoResultsError}\pysiglinewithargsret{\strong{exception }\code{errors.}\bfcode{NoResultsError}}{\emph{query}}{}
Raised when a search gives no results and it is unacceptable to return
an empty list.

\end{fulllineitems}

\index{OEISException}

\begin{fulllineitems}
\phantomsection\label{api:errors.OEISException}\pysigline{\strong{exception }\code{errors.}\bfcode{OEISException}}
Base class for PyOEIS exceptions.

\end{fulllineitems}

\index{TooManyResultsError}

\begin{fulllineitems}
\phantomsection\label{api:errors.TooManyResultsError}\pysiglinewithargsret{\strong{exception }\code{errors.}\bfcode{TooManyResultsError}}{\emph{query}}{}
Raised when too many results are found for a search for them to be
properly parsed.

\end{fulllineitems}



\chapter{Usage}
\label{usage:usage}\label{usage::doc}
The first thing you must do when using pyoeis is to initialise an
{\hyperref[api:client.OEISClient]{\code{OEISClient}}}:

\begin{Verbatim}[commandchars=\\\{\}]
\PYG{g+gp}{\PYGZgt{}\PYGZgt{}\PYGZgt{} }\PYG{k+kn}{import} \PYG{n+nn}{pyoeis}
\PYG{g+gp}{\PYGZgt{}\PYGZgt{}\PYGZgt{} }\PYG{n}{c} \PYG{o}{=} \PYG{n}{pyoeis}\PYG{o}{.}\PYG{n}{OEISClient}\PYG{p}{(}\PYG{p}{)}
\end{Verbatim}

This handles all queries to the OEIS and all methods for querying are accessed
from it. For example, say we want to access the entry for sequence A000040, the
primes:

\begin{Verbatim}[commandchars=\\\{\}]
\PYG{g+gp}{\PYGZgt{}\PYGZgt{}\PYGZgt{} }\PYG{n}{primes} \PYG{o}{=} \PYG{n}{c}\PYG{o}{.}\PYG{n}{get\PYGZus{}by\PYGZus{}id}\PYG{p}{(}\PYG{l+s}{\PYGZsq{}}\PYG{l+s}{a40}\PYG{l+s}{\PYGZsq{}}\PYG{p}{)}
\PYG{g+gp}{\PYGZgt{}\PYGZgt{}\PYGZgt{} }\PYG{n}{primes}\PYG{o}{.}\PYG{n}{name}
\PYG{g+go}{u\PYGZsq{}The prime numbers\PYGZsq{}}
\end{Verbatim}


\chapter{Indices and tables}
\label{index:indices-and-tables}\begin{itemize}
\item {} 
\emph{genindex}

\item {} 
\emph{modindex}

\item {} 
\emph{search}

\end{itemize}


\renewcommand{\indexname}{Python Module Index}
\begin{theindex}
\def\bigletter#1{{\Large\sffamily#1}\nopagebreak\vspace{1mm}}
\bigletter{e}
\item {\texttt{errors}}, \pageref{api:module-errors}
\end{theindex}

\renewcommand{\indexname}{Index}
\printindex
\end{document}
